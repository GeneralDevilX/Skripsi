\documentclass[11pt]{article}
\title{\textbf{Open Source Snake 360}}
\author{\textbf{Evelyn Wijaya - 2015730030}}
\date{}

\begin{document}
\maketitle

\section{Deskripsi}
Snake merupakan sebuah permainan yang dulu sering dimainkan di telepon genggam/handphone Nokia. Snake cukup terkenal pada masa itu, karena cara bermainya yang simpel. Kita mengendalikan pergerakan ular untuk mendapatkan makanan tanpa menabrak rintangan atau ular itu sendiri. Setiap memakan makanan, kita akan mendapat skor dan tubuh ular akan memanjang. Semakin panjang tubuh ular, maka kita akan cenderung menabrak rintangan atau tubuh ular itu sendiri. \\\\
Sekarang, sudah banyak sekali permainan Snake yang dapat dimainkan di smartphone dan web browser. Bahkan pergerakan ular juga tidak hanya 4 arah saja, tetapi sudah dapat bergerak ke segala arah. Selain itu, permainan Snake ini juga dapat dimainkan lebih dari 1 orang saja, contohnya adalah Slither.io. Meskipun sudah ada perubahan yang signifikan pada pergerakan ular, tetapi kebanyakan permainan Snake memiliki pilihan labirin yang tidak terlalu banyak sehingga pemain akan cepat bosan. \\\\
Maka dari itu, pada skripsi ini akan dibuat permainan Snake yang ularnya dapat bergerak ke segala arah dan orang lain dapat menambahkan pilihan labirin. Permainan Snake akan dibuat menggunakan HTML5/Canvas dan menambah pilihan labirin menggunakan mekanisme pull request pada Gitlab.

\section{Rumusan Masalah}
\begin{itemize}
\item Bagaimana cara membangun permainan Snake menggunakan HTML5?
\item Apa metode yang tepat untuk ular agar dapat bergerak 360 $^{\circ}$ ?
\item Apa metode yang tepat untuk menggambar tubuh ular?
\item Bagaimana cara menyimpan labirin pada file eksternal dan membuat aturanya?
\item Bagaimana cara menambah labirin menggunakan pull request pada Gitlab?
\end{itemize}

\section{Tujuan}
\begin{itemize}
\item Dapat membangun permainan Snake menggunakan HTML5
\item Dapat menentukan metode yang tepat untuk ular agar dapat bergerak 360 $^{\circ}$ ?
\item Dapat menentukan metode yang tepat untuk menggambar tubuh ular
\item Dapat menyimpan labirin pada file eksternal dan membuat aturanya
\end{itemize}

\section{Deskripsi Perangkat Lunak}
Perangkat lunak akhir yang akan dibuat memiliki fitur minimal sebagai berikut : 
\begin{itemize}
\item Pengguna dapat menggerakan ular ke segala arah dalam permainan Snake tersebut
\item Pengguna dapat menambahkan labirin menggunakan mekanisme pull request pada Gitlab yang dapat disimpan pada file eksternal
\end{itemize}

\section{Detail Pengerjaan Skripsi}
Bagian-bagian pengerjaan skripsi adalah sebagai berikut : 
\begin{enumerate}
\item Mencoba beberapa permainan Snake yang ada
\item Melakukan analisis dan menentukan objek-objek pada permainan Snake yang dibuat
\item Melakukan studi literatur 
\item Mempelajari bahasa pemrograman HTML5
\item Menentukan kelas-kelas dan membuat class diagram 
\item Merancang alogritma untuk menggambar tubuh ular, pergerakan ular dan membuat labirin
\item Mengimplentasikan keseluruhan algoritma 
\item Menambahkan labirin
\item Melakukan pengujian dan debugging
\item Menulis dokumen skripsi
\end{enumerate}

\section{Rencana Kerja}
\begin{tabular}{|c|c|c|c|c|}
			\hline
			1*&2(\%)*&3(\%)*&4(\%)*&5*\\
			\hline
			1&5&5&& \\
			\hline
			2&5&5&&\\
			\hline
			3&10&5&5&github di S2 \\ 
			\hline
			4&10&10&& \\
			\hline
			5&10&7&3&class diagram labirin di S2\\
			\hline
			6&15&10&5&merancang algoritma labirin di S2\\ 
			\hline
			7&20&10&10&pergerakan ular dan menggambar tubuh ular di S1\\
			\hline
			8&5&&5&\\
			\hline
			9&5&&5&\\
			\hline
			10&15&5&10&bab 1 dan 2 di S1\\
			\hline
			Total&100&57&43& \\
			\hline
\end{tabular}
\\\\
Keterangan(*)\\
1 : Bagian pengerjaan Skripsi(nomor disesuaikan dengan detail pengerjaan di bagian 5 \\
2 : Persentase total\\
3 : Persentase yang akan diselesaikan di Skripsi 1\\
4 : Persentase yang akan diselesaikan di Skripsi 2\\
5 : Penjelasan singkat apa yang dilakukan di S1(Skripsi 1) atau S2(Skripsi 2)\\
\end{document}