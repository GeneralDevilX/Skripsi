%_____________________________________________________________________________
%=============================================================================
% data.tex v10 (22-01-2017) dibuat oleh Lionov - T. Informatika FTIS UNPAR
%
% Perubahan pada versi 10 (22-01-2017)
%	- Penambahan overfullrule untuk memeriksa warning
%  	- perubahan mode buku menjadi 4: bimbingan, sidang(1), sidang akhir dan 
%     buku final
%	- perbaikan perintah pada beberapa bagian
%  	- perubahan pengisian tulisan "daftar isi" yang error
%  	- penghilangan lipsum dari file ini
%_____________________________________________________________________________
%=============================================================================

%=============================================================================
% 								PETUNJUK
%=============================================================================
% Ini adalah file data (data.tex)
% Masukkan ke dalam file ini, data-data yang diperlukan oleh template ini
% Cara memasukkan data dijelaskan di setiap bagian
% Data yang WAJIB dan HARUS diisi dengan baik dan benar adalah SELURUHNYA !!
% Hilangkan tanda << dan >> jika anda menemukannya
%=============================================================================

%_____________________________________________________________________________
%=============================================================================
% 								BAGIAN 0
%=============================================================================
% Entri untuk memperbaiki posisi "DAFTAR ISI" jika tidak berada di bagian 
% tengah halaman. Sayangnya setiap sistem menghasilkan posisi yang berbeda.
% Isilah dengan 0 atau 1 (e.g. \daftarIsiError{1}). 
% Pemilihan 0 atau 1 silahkan disesuaikan dengan hasil PDF yang dihasilkan.
%=============================================================================
%\daftarIsiError{0}   
\daftarIsiError{1}   
%=============================================================================

%_____________________________________________________________________________
%=============================================================================
% 								BAGIAN I
%=============================================================================
% Tambahkan package2 lain yang anda butuhkan di sini
%=============================================================================
\usepackage{booktabs} 
\usepackage{longtable}
\usepackage{amssymb}
\usepackage{todo}
\usepackage{verbatim} 		%multiline comment
\usepackage{pgfplots}
%\overfullrule=3mm % memperlihatkan overfull 
%=============================================================================

%_____________________________________________________________________________
%=============================================================================
% 								BAGIAN II
%=============================================================================
% Mode dokumen: menetukan halaman depan dari dokumen, apakah harus mengandung 
% prakata/pernyataan/abstrak dll (termasuk daftar gambar/tabel/isi) ?
% - final 		: hanya untuk buku skripsi, dicetak lengkap: judul ina/eng, 
%   			  pengesahan, pernyataan, abstrak ina/eng, untuk, kata 
%				  pengantar, daftar isi (daftar tabel dan gambar tetap 
%				  opsional dan dapat diatur), seluruh bab dan lampiran.
%				  Otomatis tidak ada nomor baris dan singlespacing
% - sidangakhir	: buku sidang akhir = buku final - (pengesahan + pernyataan +
%   			  untuk + kata pengantar)
%				  Otomatis ada nomor baris dan onehalfspacing 
% - sidang 		: untuk sidang 1, buku sidang = buku sidang akhir - (judul 
%				  eng + abstrak ina/eng)
%				  Otomatis ada nomor baris dan onehalfspacing
% - bimbingan	: untuk keperluan bimbingan, hanya terdapat bab dan lampiran
%   			  saja, bab dan lampiran yang hendak dicetak dapat ditentukan 
%				  sendiri (nomor baris dan spacing dapat diatur sendiri)
% Mode default adalah 'template' yang menghasilkan isian berwarna merah, 
% aktifkan salah satu mode di bawah ini :
%=============================================================================
%\mode{bimbingan} 		% untuk keperluan bimbingan
%\mode{sidang} 			% untuk sidang 1
\mode{sidangakhir} 	% untuk sidang 2 / sidang pada Skripsi 2(IF)
%\mode{final} 			% untuk mencetak buku skripsi 
%=============================================================================

%_____________________________________________________________________________
%=============================================================================
% 								BAGIAN III
%=============================================================================
% Line numbering: penomoran setiap baris, nomor baris otomatis di-reset ke 1
% setiap berganti halaman.
% Sudah dikonfigurasi otomatis untuk mode final (tidak ada), mode sidang (ada)
% dan mode sidangakhir (ada).
% Untuk mode bimbingan, defaultnya ada (\linenumber{yes}), jika ingin 
% dihilangkan, isi dengan "no" (i.e.: \linenumber{no})
% Catatan:
% - jika nomor baris tidak kembali ke 1 di halaman berikutnya, compile kembali
%   dokumen latex anda
% - bagian ini hanya bisa diatur di mode bimbingan
%=============================================================================
%\linenumber{no} 
\linenumber{yes}
%=============================================================================

%_____________________________________________________________________________
%=============================================================================
% 								BAGIAN IV
%=============================================================================
% Linespacing: jarak antara baris 
% - single	: otomatis jika ingin mencetak buku skripsi, opsi yang 
%			     disediakan untuk bimbingan, jika pembimbing tidak keberatan 
%			     (untuk menghemat kertas)
% - onehalf	: otomatis jika ingin mencetak dokumen untuk sidang
% - double 	: jarak yang lebih lebar lagi, jika pembimbing berniat memberi 
%             catatan yg banyak di antara baris (dianjurkan untuk bimbingan)
% Catatan: bagian ini hanya bisa diatur di mode bimbingan
%=============================================================================
\linespacing{single}
%\linespacing{onehalf}
%\linespacing{double}
%=============================================================================

%_____________________________________________________________________________
%=============================================================================
% 								BAGIAN V
%=============================================================================
% Tidak semua skripsi memuat gambar dan/atau tabel. Untuk skripsi yang tidak 
% memiliki gambar dan/atau tabel, maka tidak diperlukan Daftar Gambar dan/atau 
% Daftar Tabel. Sayangnya hal tsb sulit dilakukan secara manual karena 
% membutuhkan kedisiplinan pengguna template.  
% Jika tidak ingin menampilkan Daftar Gambar dan/atau Daftar Tabel, karena 
% tidak ada gambar atau tabel atau karena dokumen dicetak hanya untuk 
% bimbingan, isi dengan "no" (e.g. \gambar{no})
%=============================================================================
\gambar{yes}
%\gambar{no}
\tabel{yes}
%\tabel{no}  
%=============================================================================

%_____________________________________________________________________________
%=============================================================================
% 								BAGIAN VI
%=============================================================================
% Pada mode final, sidang da sidangkahir, seluruh bab yang ada di folder "Bab"
% dengan nama file bab1.tex, bab2.tex s.d. bab9.tex akan dicetak terurut, 
% apapun isi dari perintah \bab.
% Pada mode bimbingan, jika ingin:
% - mencetak seluruh bab, isi dengan 'all' (i.e. \bab{all})
% - mencetak beberapa bab saja, isi dengan angka, pisahkan dengan ',' 
%   dan bab akan dicetak terurut sesuai urutan penulisan (e.g. \bab{1,3,2}). 
% Catatan: Jika ingin menambahkan bab ke-3 s.d. ke-9, tambahkan file 
% bab3.tex, bab4.tex, dst di folder "Bab". Untuk bab ke-10 dan 
% seterusnya, harus dilakukan secara manual dengan mengubah file skripsi.tex
% Catatan: bagian ini hanya bisa diatur di mode bimbingan
%=============================================================================
\bab{all}
%=============================================================================

%_____________________________________________________________________________
%=============================================================================
% 								BAGIAN VII
%=============================================================================
% Pada mode final, sidang dan sidangkhir, seluruh lampiran yang ada di folder 
% "Lampiran" dengan nama file lampA.tex, lampB.tex s.d. lampJ.tex akan dicetak 
% terurut, apapun isi dari perintah \lampiran.
% Pada mode bimbingan, jika ingin:
% - mencetak seluruh lampiran, isi dengan 'all' (i.e. \lampiran{all})
% - mencetak beberapa lampiran saja, isi dengan huruf, pisahkan dengan ',' 
%   dan lampiran akan dicetak terurut sesuai urutan (e.g. \lampiran{A,E,C}). 
% - tidak mencetak lampiran apapun, isi dengan "none" (i.e. \lampiran{none})
% Catatan: Jika ingin menambahkan lampiran ke-C s.d. ke-I, tambahkan file 
% lampC.tex, lampD.tex, dst di folder Lampiran. Untuk lampiran ke-J dan 
% seterusnya, harus dilakukan secara manual dengan mengubah file skripsi.tex
% Catatan: bagian ini hanya bisa diatur di mode bimbingan
%=============================================================================
\lampiran{all}
%=============================================================================

%_____________________________________________________________________________
%=============================================================================
% 								BAGIAN VIII
%=============================================================================
% Data diri dan skripsi/tugas akhir
% - namanpm		: Nama dan NPM anda, penggunaan huruf besar untuk nama harus 
%				  benar dan gunakan 10 digit npm UNPAR, PASTIKAN BAHWA 
%				  BENAR !!! (e.g. \namanpm{Jane Doe}{1992710001}
% - judul 		: Dalam bahasa Indonesia, perhatikan penggunaan huruf besar, 
%				  judul tidak menggunakan huruf besar seluruhnya !!! 
% - tanggal 	: isi dengan {tangga}{bulan}{tahun} dalam angka numerik, 
%				  jangan menuliskan kata (e.g. AGUSTUS) dalam isian bulan.
%			  	  Tanggal ini adalah tanggal dimana anda akan melaksanakan 
%				  sidang ujian akhir skripsi/tugas akhir
% - pembimbing	: pembimbing anda, lihat daftar dosen di file dosen.tex
%				  jika pembimbing hanya 1, kosongkan parameter kedua 
%				  (e.g. \pembimbing{\JND}{} ), \JND adalah kode dosen
% - penguji 	: para penguji anda, lihat daftar dosen di file dosen.tex
%				  (e.g. \penguji{\JHD}{\JCD} )
% !!Lihat singkatan pembimbing dan penguji anda di file dosen.tex!!
% Petunjuk: hilangkan tanda << & >>, dan isi sesuai dengan data anda
%=============================================================================
\namanpm{Evelyn Wijaya}{2015730030}
\tanggal{<<tanggal>>}{<<bulan>>}{2019}
\pembimbing{<<pembimbing utama/1>>}{}    
\penguji{<<penguji 1>>}{<<penguji 2>>} 
%=============================================================================

%_____________________________________________________________________________
%=============================================================================
% 								BAGIAN IX
%=============================================================================
% Judul dan title : judul bhs indonesia dan inggris
% - judulINA: judul dalam bahasa indonesia
% - judulENG: title in english
% Petunjuk: 
% - hilangkan tanda << & >>, dan isi sesuai dengan data anda
% - langsung mulai setelah '{' awal, jangan mulai menulis di baris bawahnya
% - gunakan \texorpdfstring{\\}{} untuk pindah ke baris baru
% - judul TIDAK ditulis dengan menggunakan huruf besar seluruhnya !!
%=============================================================================
\judulINA{Open Source Snake 360}
\judulENG{Open Source Snake 360}
%_____________________________________________________________________________
%=============================================================================
% 								BAGIAN X
%=============================================================================
% Abstrak dan abstract : abstrak bhs indonesia dan inggris
% - abstrakINA: abstrak bahasa indonesia
% - abstrakENG: abstract in english 
% Petunjuk: 
% - hilangkan tanda << & >>, dan isi sesuai dengan data anda
% - langsung mulai setelah '{' awal, jangan mulai menulis di baris bawahnya
%=============================================================================
\abstrakINA{Penelitian ini membahas mengenai pembangunan permainan Open Source Snake 360. Permainan ini dibuat berdasarkan acuan dari permainan \textit{Snake} yang sudah ada. Permainan \textit{Snake} adalah permainan mengontrol gerakan ular untuk mendapatkan makanan yang tersebar di labirin. Setiap ular memakan makanan, pemain akan mendapatkan skor. Pada permainan ini, pemain harus mengontrol ular untuk mendapatkan makanan sebanyak-banyaknya tanpa menabrak dinding labirin atau dirinya sendiri. Permainan ini sudah umum dimainkan pada \textit{web browser} dan beberapa perangkat. Umumya pada permainan Snake, ular hanya dapat bergerak ke atas, ke bawah, ke kiri dan ke kanan saja. Selain itu labirin yang disediakan terbatas. \\

HTML(\textit{Hyper Text Markup Language}) merupakan bahasa markah yang digunakan untuk membuat halaman web. HTML5 merupakan HTML versi terbaru. HTML5 memiliki beberapa elemen baru, salah satunya adalah HTML5 Canvas. HTML5 Canvas adalah tempat untuk menggambar \textit{pixel-pixel} yang dapat ditulis menggunakan bahasa pemrograman \textit{JavaScript}. \textit{Javascript} merupakan bahasa tingkat tinggi yang digunakan untuk membuat halaman \textit{web} menjadi lebih interaktif dan \textit{jQuery} merupakan library milik \textit{Javascript} yang kaya fitur. \textit{GitHub} adalah layanan hosting bersama untuk proyek pengembangan perangkat lunak yang menggunakan sistem \textit{version control} yaitu \textit{Git}. Dengan adanya \textit{GitHub}, \textit{programmer} dapat mengetahui perubahan yang pada \textit{repository} tersebut.\\

Open Source Snake 360 adalah sebuah permainan yang dibuat menggunakan HTML5 dan \textit{Javascript}. \textit{jQuery} digunakan untuk mengakses labirin dan memuat labirin dari server. Pada permainan ini, ular sudah dapat bergerak ke segala arah dan orang lain dapat menambahkan labirin buatan sendiri. Orang lain dapat menambahkan labirin buatan sendiri dengan menggunakan \textit{pull request} pada \textit{GitHub}. Permainan ini juga sudah dapat memuat labirin-labirin yang dibuat oleh orang lain.}

\abstrakENG{This study discusses the construction of the Open Source Snake 360 game. This game is based on the references of the existing Snake game. Snake game is a game in which players control the movement of snakes to get food scattered in the maze. Every snake eats food, the player gets a score. In this game, players must control the snake to get as much food as possible without crashing into the wall of the labyrinth or himself. This game is commonly played on web browser and some devices. Generally speaking in Snake game, the snake can only move up, down, left and right. In addition, the labyrinth provided is limited.\\

HTML(Hyper Text Markup Language) is the marking language used to create web pages. HTML5 is the latest version of HTML. HTML5 has several new elements, one of which is HTML5 Canvas. HTML5 Canvas is a place to draw pixels that can be written using the Javascript programming language. Javasciprt is a high-level language used to make web pages more interactive and jQuery is a feature-rich Javascript library. GitHub is a shared hosting service for software development projects that uses the version control system, Git. With GitHub, the programmer can find out the changes in the repository.\\

Open Source Snake 360 is a game created using HTML5 and Javascript. jQuery is used to access and load the maze from the server. In this game, snake can move in any direction and others can add homemade mazes. Other people can add homemade mazes by using pull request on GitHub. This game can also load mazes made by other people.}
%=============================================================================

%_____________________________________________________________________________
%=============================================================================
% 								BAGIAN XI
%=============================================================================
% Kata-kata kunci dan keywords : diletakkan di bawah abstrak (ina dan eng)
% - kunciINA: kata-kata kunci dalam bahasa indonesia
% - kunciENG: keywords in english
% Petunjuk: hilangkan tanda << & >>, dan isi sesuai dengan data anda.
%=============================================================================
\kunciINA{Snake Game, HTML, Javascript, jQuery, GitHub}
\kunciENG{Snake Game, HTML, Javascript, jQuery, GitHub}
%=============================================================================

%_____________________________________________________________________________
%=============================================================================
% 								BAGIAN XII
%=============================================================================
% Persembahan : kepada siapa anda mempersembahkan skripsi ini ...
% Petunjuk: hilangkan tanda << & >>, dan isi sesuai dengan data anda.
%=============================================================================
\untuk{Dipersembahkan kepada orang tua}
%=============================================================================

%_____________________________________________________________________________
%=============================================================================
% 								BAGIAN XIII
%=============================================================================
% Kata Pengantar: tempat anda menuliskan kata pengantar dan ucapan terima 
% kasih kepada yang telah membantu anda bla bla bla ....  
% Petunjuk: hilangkan tanda << & >>, dan isi sesuai dengan data anda.
%=============================================================================
\prakata{<<Tuliskan kata pengantar dari anda di sini \ldots>>} 
%=============================================================================

%_____________________________________________________________________________
%=============================================================================
% 								BAGIAN XIV
%=============================================================================
% Tambahkan hyphen (pemenggalan kata) yang anda butuhkan di sini 
%=============================================================================
\hyphenation{ma-te-ma-ti-ka}
\hyphenation{fi-si-ka}
\hyphenation{tek-nik}
\hyphenation{in-for-ma-ti-ka}
%=============================================================================

%_____________________________________________________________________________
%=============================================================================
% 								BAGIAN XV
%=============================================================================
% Tambahkan perintah yang anda buat sendiri di sini 
%=============================================================================
\renewcommand{\vtemplateauthor}{lionov}
\pgfplotsset{compat=newest}
%=============================================================================
