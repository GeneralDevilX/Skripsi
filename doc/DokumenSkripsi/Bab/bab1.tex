%versi 2 (8-10-2016) 
\chapter{Pendahuluan}
\label{chap:intro}
   
\section{Latar Belakang}
\label{sec:label}

\textit{Snake} merupakan sebuah permainan yang pertama kali dibuat oleh Peter Trefonas pada tahun 1978. Konsep \textit{Snake} berasal dari permainan arkade yaitu \textit{Blockade}. Awalnya \textit{Snake} hanya dapat dimainkan pada komputer pribadi. Namun pada tahun 1997, \textit{Snake} dapat dimainkan pada telepon genggam \textit{Nokia}\footnote{https://en.wikipedia.org/wiki/Snake\_(video\_ game\_ genre)}. Cara bermain \textit{Snake} adalah pemain menggerakan ular pada sebuah labirin. Ular tersebut harus mendapatkan makanan sebanyak-banyaknya tanpa menabrak dinding atau ular itu sendiri. Setiap memakan makanan, tubuh ular akan memanjang dan pemain akan semakin sulit untuk menggerakan ular tersebut dengan bebas karena tubuh ular semakin lama akan menutupi labirin tersebut. \\

HTML(\textit{Hyper Text Markup Language}) adalah sebuah bahasa markah yang digunakan untuk membuat halaman web. HTML5 merupakan HTML versi 5 yang terbaru dan penerus dari HTML4, XHTML1, dan DOM level 2 HTML. HTML5 memiliki beberapa elemen baru, salah satunya adalah HTML5 Canvas. HTML5 Canvas adalah tempat untuk menggambar \textit{pixel-pixel} yang dapat ditulis menggunakan bahasa pemrograman \textit{JavaScript}. \textit{Javascript} adalah bahasa pemrograman tingkat tinggi yang digunakan untuk membuat halaman web menjadi lebih interaktif. \textit{GitHub} adalah layanan \textit{web hosting} bersama untuk proyek pengembangan perangkat lunak yang menggunakan sistem \textit{version control} yaitu \textit{Git}. Dengan adanya \textit{Github}, \textit{programmer} dapat mengetahui perubahan yang pada \textit{repository} tersebut. \\

Pada permainan \textit{Snake}, umumnya pergerakan ular hanya atas, bawah, kiri, dan kanan saja. Pada skripsi ini, penulis akan membuat permainan \textit{Snake} yang ularnya dapat bergerak ke segala arah dan orang lain dapat menambahkan labirin menggunakan mekanisme \textit{pull request Github}. Dengan begitu, orang lain dapat menambahkan labirin sesuai dengan keinginanya dan pemain tidak akan cepat bosan karena labirin yang disediakan cukup banyak dan variatif.

\section{Rumusan Masalah}
\label{sec:rumusan}
Rumusan dari masalah yang akan dibahas pada skripsi ini adalah sebagai berikut:
\begin{itemize}
	\item Bagaimana membangun permainan \textit{Snake} menggunakan HTML5?
	\item Bagaimana cara menyimpan labirin pada file eksternal?
	\item Bagaimana cara menggunakan \textit{pull request} pada \textit{Github} agar orang lain dapat menambahkan labirin?
\end{itemize}


\section{Tujuan}
\label{sec:tujuan}
Tujuan-tujuan yang hendak dicapai melalui penulisan skripsi ini adalah sebagai berikut:
\begin{itemize}
	\item Dapat membangun permainan \textit{Snake} menggunakan HTML5.
	\item Dapat menyimpan labirin pada file eksternal.
	\item Dapat menggunakan \textit{pull request} pada \textit{Github} agar orang lain dapat menambahkan labirin.
\end{itemize}


\section{Batasan Masalah}
\label{sec:batasan}
Beberapa batasan yang dibuat terkait dengan pengerjaan skripsi ini adalah sebagai berikut:
\begin{itemize}
	\item Permainan ini hanya dapat dimainkan menggunakan \textit{web browser} pada komputer.
	\item \textit{Web browser} yang digunakan sudah mendukung HTML5 Canvas.
\end{itemize}


\section{Metodologi}
\label{sec:metlit}
Metodologi pada penelitian ini adalah sebagai berikut:
\begin{enumerate}
	\item Melakukan studi literatur tentang HTML5, \textit{JavaScript}, dan \textit{Git}.
	\item Melakukan analisis dan menentukan objek-objek pada \textit{Snake}.
	\item Merancang algoritma untuk menggambar tubuh ular, pergerakan ular dan membuat labirin.
	\item Mengimplementasikan keseluruhan algoritma.
	\item Menambahkan labirin menggunakan \textit{pull request} pada \textit{Github}.
	\item Melakukan pengujian.
	\item Melakukan penarikan kesimpulan. 
\end{enumerate}


\section{Sistematika Pembahasan}
\label{sec:sispem}
Sistematikan penulisan setiap bab pada penelitian ini adalah sebagai berikut:
\begin{enumerate}
	\item Bab 1 berisikan latar belakang, rumusan masalah, tujuan, batasan masalah, metodologi, dan sistematika pembahasan dari penelitian yang dilakukan.
	\item Bab 2 berisikan dasar-dasar teori yang menunjang penelitian ini. Teori yang digunakan adalah: pengertian \textit{Snake}, HTML5 Canvas, \textit{Javascript}, dan \textit{Git}.
	\item Bab 3 berisikan analisis sistem yang dibangun, analisis sistem yang dibangun dan analisis berorientasi objek.
\end{enumerate}