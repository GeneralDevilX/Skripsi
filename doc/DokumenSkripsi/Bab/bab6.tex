\chapter{Kesimpulan dan Saran}
\label{chap:kesimpulansaran}

Pada bab ini berisi kesimpulan dari pembangunan permainan dan saran untuk pengembangan permainan ini.

\section{Kesimpulan}
Dari hasil pembangunan permainan Open Source Snake 360, dapat diambil beberapa kesimpulan, diantaranya adalah: 

\begin{enumerate}
	\item Pembangunan permainan Open Source Snake 360 menggunakan HTML5 Canvas sudah berhasil kecuali posisi apel pada labirin dan pergerakan berbelok ular pada smartphone belum sesuai dengan yang diharapkan. Pada pengujian fungsional, pengujian menekan tombol '\textit{Play Game}' dan pengujian pergerakan berbelok ular pada smartphone belum berhasil.
	\item Menambahkan labirin menggunakan \textit{pull request GitHub} berhasil dilakukan. Hal ini dapat dilihat berdasarkan pengujian eksperimental yang sudah dilakukan. Semua penguji berhasil menambahkan labirin buatan sendiri menggunakan \textit{pull request}.
	\item Menyimpan labirin pada \textit{file} eksternal berhasil dilakukan. Hal ini dapat dilihat pada pengujian fungsional yang sudah dilakukan. Pada pengujian menekan tombol '\textit{Play Game}', labirin sudah disimpan pada \textit{file} eksternal serta sudah dapat dimuat dan digambar.
	\item Labirin cukup sulit untuk dibuat. Hal ini dapat dilihat berdasarkan pengujian eksperimental yang sudah dilakukan. 2 dari 5 penguji tidak berhasil membuat labirin dengan benar.
\end{enumerate} 

\section{Saran}
Berdasarkan kesimpulan yang telah dipaparkan, terdapat beberapa saran yang dapat digunakan untuk pengembangan permainan ini. Berikut adalah saran-saran yang ada:

\begin{enumerate}
	\item \textit{Format} labirin harus dibuat lebih mudah untuk mengurangi kesalahan pada pembuatan labirin terutama pada memposisikan ular. Seharusnya, posisi ular tidak perlu ditulis dalam koordinat melainkan dituliskan langsung menggunakan sebuah simbol pada labirin.
	\item Sebelum memulai permainan, sebaiknya terdapat preview untuk setiap labirin, sehingga pemain tidak perlu memulai permainan terlebih dahulu untuk melihat labirin yang dipilih.
\end{enumerate}