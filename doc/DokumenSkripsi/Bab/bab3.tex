\chapter{Analisis}
\label{chap:analisis}

\section{Analisis Permainan \textit{Snake} yang Sudah Ada}
Permainan \textit{Snake} yang akan dianalisis adalah \textit{Slither.io} dan Snake pada telepon genggam Nokia. Slither.io adalah permainan \textit{web} yang dapat dimainkan oleh lebih dari 1 pemain(\textit{multiplayer}). Cara bermainya mirip seperti permainan \textit{Snake} pada umumnya yaitu ular harus memakan makanan untuk mendapatkan skor. Dalam permainan ini, setiap pemain berkompetisi untuk menjadi pemain terbaik dengan cara mendapatkan skor sebanyak-banyaknya. Pemain akan kalah apabila ular milik pemain menabrak ular milik pemain lain.\\

Snake pada telepon genggam Nokia hanya dapat dimainkan oleh 1 pemain. Dalam permainan ini, ular harus mendapatkan skor sebanyak-banyaknya dengan memakan makanan. Setiap memakan makanan, skor akan bertambah sebanyak 1 poin. Pemain akan kalah apabila ular menabrak dinding labirin.

\subsection{Ular dan Makanan}
Ular pada \textit{Slither.io} dibentuk dengan menggunakan sekumpulan lingkaran yang saling berdempetan satu sama lain seperti pada Gambar~\ref{fig:slitherUlar}. Bagian kepala pada ular ditandai menggunakan sepasang mata. Ketika memakan makanan, tubuh ular akan memanjang dengan menambahkan sebuah lingkaran pada bagian ekor ular. Setiap memulai permainan, tubuh ular akan memiliki warna yang ditentukan secara acak.\\

Makanan pada \textit{Slither.io} berbentuk lingkaran. Makanan ini ada yang berukuran besar dan ada yang berukuran kecil. Makanan ini tersebar pada labirin, jumlahnya sangat banyak dan warnanya bermacam-macam. Gambar~\ref{fig:slitherMakanan} merupakan sekumpulan makanan yang terdapat pada labirin. Setiap makanan akan menambah skor sebanyak 1 poin.

\begin{figure}[H]
	\centering  
	\includegraphics[scale=0.7]{slitherUlar}  
	\caption[Ular pada \textit{Silther.io}]{Ular pada \textit{Silther.io}}
	\label{fig:slitherUlar} 
\end{figure}

\begin{figure}[H]
	\centering  
	\includegraphics[scale=0.7]{slitherMakanan}  
	\caption[Makanan pada \textit{Slither.io}]{Makanan pada \textit{Slither.io}}
	\label{fig:slitherMakanan} 
\end{figure}

Ular pada Snake Nokia dibuat seperti permainan 8 bit yang terdiri dari pixel-pixel seperti pada Gambar~\ref{fig:snakeNokia}. Pada permainan ini apabila kepala ular sudah dekat dengan makanan, maka kepala ular akan terlihat sedang membuka mulutnya. Makanan yang terdapat pada permainan ini ada 2 macam yaitu makanan biasa dan makanan bonus seperti yang terlihat pada Gambar~\ref{fig:snakeFood}. Makanan biasa memiliki skor 1 poin dan makanan bonus memiliki skor 10 poin. Makanan bonus muncul secara acak dan memiliki batas waktu untuk berada pada labirin. Makanan bonus tidak hanya menambah skor lebih banyak saja tetapi makanan ini dapat membuat tubuh ular lebih panjang dibandingkan dengan memakan makanan biasa.

\begin{figure}[H]
	\centering  
	\includegraphics[scale=1]{snakeNokia}  
	\caption[Ular pada \textit{Snake Nokia}]{Ular pada \textit{Snake Nokia}}
	\label{fig:snakeNokia} 
\end{figure}

\begin{figure}[H]
	\centering  
	\includegraphics[scale=0.7]{snakeFood}  
	\caption[Makanan biasa(A) dan makanan bonus(B) pada \textit{Snake Nokia}]{Makanan biasa(A) dan makanan bonus(B) pada \textit{Snake Nokia}}
	\label{fig:snakeFood} 
\end{figure}

\subsection{Pergerakan Ular}
Ular pada \textit{Slither.io} digerakan dengan menggunakan keyboard dan \textit{mouse}. Tombol ke kiri akan membuat ular bergerak berlawanan arah jarum jam dan tombol ke kanan akan membuat ular bergerak searah jarum jam. Semakin lama tombol ditekan, maka ular akan berbelok lebih cepat. Kursor pada \textit{mouse} membuat ular bergerak ke arah posisi kursor tersebut. Ular dapat melaju dengan cepat(\textit{speed up}) dengan menekan tombol \textit{mouse} kiri seperti yang terdapat pada Gambar~\ref{fig:slitherSpeed}. Ketika ular sedang melaju dengan cepat, total skor yang didapat akan berkurang. 

\begin{figure}[H]
	\centering  
	\includegraphics[scale=0.7]{slitherSpeed}  
	\caption[Ular sedang melaju dengan cepat(\textit{speed up})]{Ular sedang melaju dengan cepat(\textit{speed up})}
	\label{fig:slitherSpeed} 
\end{figure}

Ular pada Snake Nokia hanya dapat bergerak ke atas, ke bawah, ke kiri dan ke kanan. Ular dapat digerakan menggunakan tombol angka pada telepon genggam Nokia yaitu tombol 8 untuk bergerak ke atas, tombol 4 untuk bergerak ke kiri, tombol 6 untuk bergerak ke kanan dan tombol 2 untuk bergerak ke bawah. Kecepatan ular juga dapat dipilih. Semakin tinggi tingkat, maka ular akan bergerak semakin cepat.

\subsection{Labirin}
Labirin pada \textit{Slither.io} hanya ada 1 saja. Labirin ini berbentuk lingkaran yang sisinya merupakan dinding. Apabila ular menabrak dinding labirin, maka permainan akan berakhir. Labirin ini cukup besar sehingga sangat kecil kemungkinan ular untuk menabrak dinding labirin. Gambar~\ref{fig:slitherLabirin} menunjukan peta labirin pada \textit{Slither.io}. Pada peta labirin tersebut terdapat sekumpulan titik bewarna abu-abu yang merepresentasikan makanan.

\begin{figure}[H]
	\centering  
	\includegraphics[scale=1]{slitherLabirin}  
	\caption[Peta labirin pada \textit{Slither.io}]{Peta labirin pada \textit{Slither.io}}
	\label{fig:slitherLabirin} 
\end{figure}

Labirin pada Snake Nokia lebih bervariasi dibandingkan dengan Slither.io. Pada permainan ini pemain dapat memilih labirin yang tersedia. Labirin dengan level yang lebih tinggi akan memiliki labirin yang rumit dan memiliki lebih banyak dinding. 

\section{Analisis Sistem yang Dibangun}
Permainan \textit{Snake} 360 yang akan dibangun memiliki cara bermain yang mirip seperti permainan Snake pada umumnya. Perbedaan antara \textit{Snake} 360 dengan permainan \textit{Snake} pada umumnya adalah \textit{Snake} 360 dapat menambahkan level dan labirin sendiri. 

\subsection{Menentukan Besar \textit{Canvas}}
Pada permainan Open Source Snake 360 ini, pemain dapat memainkan permainan tersebut di smartphone dan browser pada desktop. Hal ini akan memunculkan sebuah kesulitan yaitu menentukan dimensi canvas yang cocok apabila permainan tersebut dimainkan pada smartphone dan browser pada desktop. Apabila besar canvas disesuaikan dengan besar layar pada desktop, maka objek yang ditampilkan(ular dan apel) akan menjadi sangat kecil. Selain itu, apabila besar canvas disesuaikan dengan besar layar smartphone, maka besar canvas akan terlihat terlalu tinggi bila dilihat pada desktop. Maka dari itu, besar canvas tidak sesuai jika bentuk canvas adalah persegi panjang. Bentuk canvas yang sesuai adalah bentuk persegi dengan dimensi 600px x 600px. Dimensi ini sesuai jika permainan ini dimainkan pada smartphone dan desktop. Selain menentukan dimensi, besar objek yang ada pada canvas juga harus diperbesar agar objek-objek pada canvas tidak terlihat kecil bagi pemain bermain menggunakan smartphone.

\subsection{Menggambar Ular dan Apel}
Tubuh ular dibuat menggunakan sekumpulan \textit{line}/garis pendek. Setiap bagian tubuh ular memiliki panjang sebesar 1 \textit{pixel} dan lebar tubuhnya sebesar 5 \textit{pixel}. Bagian tubuh ular dibuat pendek untuk memudahkan pengecekan jika terjadi ular menabrak tubuhnya sendiri. Untuk lebar ular, disesuaikan dengan besar apel yaitu 10 \textit{pixel}. Setiap bagian tubuh ular memiliki koordinat masing-masing. Koordinat setiap bagian tubuh disimpan pada sebuah \textit{array} agar menggambar ular menjadi lebih mudah. Dalam tahap ini, tubuh ular masih berupa sekumpulan titik-titik yang merupakan koordinat bagian tubuh ular seperti pada Gambar~\ref{fig:titikUlar}. Algoritma untuk menggambar ular adalah dengan mengambil koordinat bagian tubuh ular mulai dari elemen \textit{array} paling pertama(arr[0]) dan elemen \textit{array} selanjutnya(arr[1]) lalu buat garis yang \textit{start point}nya adalah elemen pertama(arr[0]) dan \textit{end point}nya adalah elemen \textit{array} kedua(arr[1]). Setelah itu ambil koordinat elemen \textit{array} yang merupakan \textit{end point} pada garis sebelumnya(arr[1]) dengan elemen \textit{array} selanjutnya(arr[2]) dan gambar garisnya. Lakukan hal tersebut sampai \textit{end point} garis mencapai elemen \textit{array} paling akhir. Setelah digambar maka ular akan terlihat seperti Gambar~\ref{fig:garisUlar}.

\begin{figure}[H]
	\centering  
	\includegraphics[scale=0.7]{titikUlar}  
	\caption[Koordinat bagian tubuh ular pada \textit{array}]{Koordinat bagian tubuh ular pada \textit{array}}
	\label{fig:titikUlar} 
\end{figure}

\begin{figure}[H]
	\centering  
	\includegraphics[scale=0.7]{garisUlar}  
	\caption[Tubuh ular setelah digambar menggunakan garis]{Tubuh ular setelah digambar menggunakan garis}
	\label{fig:garisUlar} 
\end{figure}

Untuk membuat apel digunakan \textit{quadratic B\'ezier curve}. Kurva ini digunakan untuk membuat bagian-bagian apel yang melengkung. Bagian tersebut ditandai dengan lingkaran bewarna merah seperti yang ditunjukan pada Gambar~\ref{fig:apel}(gambar diambil dari pinterest. Link:https://www.pinterest.com/pin/690317449105509454/).

\begin{figure}[H]
	\centering  
	\includegraphics[scale=0.5]{apel}  
	\caption[Bagian pada apel(lingkaran merah) yang akan dibuat menggunakan kurva]{Bagian pada apel(lingkaran merah) yang akan dibuat menggunakan kurva}
	\label{fig:apel} 
\end{figure}

Pertama, tentukan besar apel yang ingin dibuat. Dalam permainan ini besar apel yang dibuat adalah 10 pixel. Besar apel dibuat lebih besar dari lebar ular karena jika besar apel sama dengan lebar ular, besar apel terlihat sangat kecil. Selain itu, apel ini digambar pada layout yang berbentuk persegi. Layout persegi ini juga dapat mempermudah penggambaran apel. Karena menggunakan layout persegi, maka origin terletak pada titik sudut di sebelah kiri atas. Setelah itu, gambar setiap bagian apel. Bagian apel dibagi menjadi 4 seperti pada Gambar~\ref{fig:apel2} sehingga besar setiap bagian apel tersebut adalah 5 pixel. 

\begin{figure}[H]
	\centering  
	\includegraphics[scale=0.4]{apel2}  
	\caption[Pembagian gambar apel dengan layout persegi beserta ukuran pada setiap bagian]{Pembagian gambar apel dengan layout persegi beserta ukuran pada setiap bagian}
	\label{fig:apel2} 
\end{figure}

Gambar bagian atas apel terlebih dahulu. Gunakan method moveTo() untuk menentukan titik mulainya. Titik mulainya terletak pada bagian tengah atas apel yang melengkung ke dalam. Dari titik itu, buat kurva yang control pointnya adalah titik ujung layout persegi. Jika ingin menggambar bagian kiri apel terlebih dahulu maka control pointnya adalah titik ujung kiri layout tersebut. Setelah itu, tentukan end point kurva tersebut. Pada Gambar~\ref{fig:apel3} terdapat start point, control point dan end point untuk membuat bagian sisi kiri atas apel. Sesudah itu, buatlah bagian bawah apel. Caranya sama seperti sebelumnya namun control pointnya dan end pointnya berbeda. Posisi control pointnya sedikit menjorok ke dalam dan posisi end pointnya terdapat di tengah bawah seperti pada Gambar~\ref{fig:apel4}. Start point tidak perlu diatur lagi, karena start pointnya sudah tergantikan dengan posisi end point pada kurva sebelumnya. Sampai pada bagian ini, bagian kiri apel sudah selesai dibuat. Untuk membuat bagian kanan apel, caranya sama seperti membuat bagian kiri apel. Karena bagian kiri apel simetris dengan bagian kanan apel, maka hanya perlu mengubah control point dan end pointnya saja. Dengan memanfaatkan bentuk simetris dari apel, maka jarak antara control point dan end point pada bagian kiri apel dengan batasan tengah sama dengan jarak antara control point dan end point dengan batas tengah pada bagian kanan apel. 

\begin{figure}[H]
	\centering  
	\includegraphics[scale=0.4]{apel3}  
	\caption[Start point, control point dan end point untuk menggambar apel bagian kiri atas]{Start point, control point dan end point untuk menggambar apel bagian kiri atas}
	\label{fig:apel3} 
\end{figure}

\begin{figure}[H]
	\centering  
	\includegraphics[scale=0.4]{apel4}  
	\caption[Start point, control point dan end point untuk menggambar apel bagian kiri bawah]{Start point, control point dan end point untuk menggambar apel bagian kiri bawah}
	\label{fig:apel4} 
\end{figure}

\subsection{Pergerakan Ular}
Untuk membuat ular bergerak maju, dilakukan penambahan kepala dan pembuangan ekor secara bersamaan ketika ular sedang bergerak maju. Ilustrasinya dapat dilihat pada Gambar~\ref{fig:snakeMoveForward}. Untuk membuat ular bergerak dengan menggunakan cara pada Gambar~\ref{fig:snakeMoveForward}, algoritmanya adalah sebagai berikut : Pertama, semua elemen array akan dishift/digeser dan elemen pertama akan digantikan dengan koordinat yang baru. Setelah itu dilakukan pengecekan apakah panjang tubuh ular lebih besar dari jumlah elemen array tubuh ular. Jika benar, maka tidak dilakukan pembuangan elemen terakhir dan jika salah, maka tidak akan dilakukan apa-apa. 

\begin{figure}[H]
	\centering  
	\includegraphics[scale=0.5]{snakeMoveForward}  
	\caption[Ilustrasi ular sebelum bergerak maju(A) dan setelah bergerak maju(B)]{Ilustrasi ular sebelum bergerak maju(A) dan setelah bergerak maju(B)}
	\label{fig:snakeMoveForward} 
\end{figure}

Kecepatan ular pada permainan ini adalah 1 sampai 5 \textit{pixel per frame}. Kecepatan maksimal ular tidak boleh melebihi lebar tubuh ular. Jika kecepatanya melebihi lebar ular, maka ketika terjadi tabrakan dengan tubuhnya sendiri, kepala ular tidak akan bertabrakan dengan tubuhnya. Kepala ular akan terlihat seolah-olah melompati tubuhnya sendiri. Dalam permainan ini, kecepatan ular adalah 2 \textit{pixel per frame}, karena dengan kecepatan 1 \textit{pixel per frame}, ular terlihat bergerak lebih lambat.\\

Ular dapat berbelok dengan menggunakan tombol pada \textit{keyboard}. Tombol ke kiri akan membuat ular bergerak melawan arah jarum jam dan tombol ke kanan akan membuat ular akan bergerak searah jarum jam. Pada permainan yang akan dibuat ini, digunakan sudut sebagai nilai untuk membuat ular dapat bergerak 360$^\circ$. Jika menekan tombol ke kiri maka sudut akan berkurang dan jika menekan tombol ke kanan maka sudut akan bertambah. Ketika menambahkan dan mengurangi sudut, perlu dilakukan pengecekan apabila nilai sudut valid atau tidak. Karena nilai sudut yang valid adalah antara nilai 0 sampai 360, maka apabila nilai sudut kurang dari 0, ubahlah sudut tersebut menjadi 360 dan apabila nilai sudut lebih besar dari 360, ubahlah nilai sudut tersebut menjadi 0. Dibutuhkan rumus trigonometri untuk menentukan posisi kepala ular. Untuk menghitung posisi koordinat x, digunakan \textit{sinus} sedangkan untuk menghitung posisi koordinat y menggunakan \textit{cosinus}. Jadi koordinat x dan y pada kepala ular akan ditambahkan dengan hasil perhitungan \textit{sinus} dan \textit{cosinus}.

\subsection{Mengacak posisi apel}
Posisi apel akan diacak di daerah canvas. Untuk mengacak posisi apel, digunakan fungsi Math.random(). Nilai yang akan diacak adalah posisi x dan y dari apel. Hasil dari fungsi Math.random() akan dikalikan dengan lebar canvas untuk mendapatkan nilai x dan dikalikan dengan tinggi canvas untuk mendapatkan nilai y. Karena apel ini dibuat dengan menggunakan layout persegi, maka posisi x dan y pada apel terletak di titik sudut kiri atas. Hal ini akan memungkinkan gambar apel akan terpotong seperti yang terlihat pada Gambar~\ref{fig:apelTerpotong}. Misal, besar canvas adalah 500 x 500 dan besar apel adalah 10 dan mendapatkan posisi apel adalah (495,10). Posisi x apel ditambah dengan besar apel hasilnya akan melebihi besar canvas sehingga membuat sebagian gambar apel terlihat terpotong. Maka dari itu, lebar dan tinggi canvas yang dikalikan dengan bilangan acak, akan dikurangi sebesar ukuran apel tersebut.  Nilai yang dihasilkan adalah nilai yang bertipe float sedangkan posisi x dan y pada apel membutuhkan input bilangan bulat. Untuk mendapatkan bilangan bulat tersebut, nilai yang sudah dihitung tadi dibulatkan ke bawah. Mengacak posisi apel tidak hanya mengacak posisi pada canvas saja, tetapi harus mengecek apakah posisi apel tersebut tidak bertabrakan dengan tubuh ular atau dinding labirin. 

\begin{figure}[H]
	\centering  
	\includegraphics[scale=0.5]{apelTerpotong}  
	\caption[Gambar apel yang terpotong sesudah mengacak posisi apel]{Gambar apel yang terpotong sesudah mengacak posisi apel}
	\label{fig:apelTerpotong} 
\end{figure}

\subsection{Menggambar Labirin}
Labirin pada permainan ini dibuat pada file text. Permainan ini dapat mengambil file text labirin sesuai dengan yang pemain inginkan. Pada file text, daerah yang merupakan dinding labirin ditulis dengan menggunakan simbol '\#' sedangkan untuk daerah yang bukan merupakan dinding labirin ditulis dengan menggunakan simbol '-'. Sebuah simbol merepresentasikan besar dinding. Jika pada file tersebut terdapat text '\#-\#', itu artinya mengammbar dinding, tidak menggambar dinding dan menggambar dinding lagi. Hasilnya dapat dilihat pada Gambar~\ref{fig:gambarDinding}. Besar canvas untuk permainan ini adalah 500 pixel x 500 pixel dan besar dinding labirin sebesar 5 pixel. Besar dinding tidak boleh lebih kecil dari lebar tubuh ular. Apabila besar dinding lebih kecil dari ular, ular tidak akan dapat melewati jalur yang diapit oleh 2 buah dinding seperti yang terlihat pada Gambar~\ref{fig:ularDinding}. Oleh karena itu, jumlah baris pada file text akan disamakan dengan jumlah kolomnya Sesuai dengan besar canvas dan besar dinding labirin, maka dapat ditentukan bahwa sebuah file text memiliki 100 baris. Setiap baris akan menggambar garis yang memiliki lebar 5 pixel. Untuk menggambar dinding secara horizontal, maka hanya menggambar garis dengan panjang 5 pixel dari titik awal ke titik akhir. Sebagai contoh, apabila karakter pada baris pertama dan kolom pertama adalah '\#', maka dinding akan digambar pada canvas dari titik(0,0) sampai titik(5,0) dengan lebar dinding 5 pixel. Level pada labirin dapat ditentukan berdasarkan kerumitan labirin. Labirin yang memiliki dinding yang banyak dan kompleks akan mendapatkan level yang lebih tinggi dibandingkan dengan labirin yang memiliki sedikit dinding dan lebih simpel. 

\begin{figure}[H]
	\centering  
	\includegraphics[scale=0.5]{gambarDinding}  
	\caption[Menggambar dinding menggunakan simbol pada file text]{Menggambar dinding menggunakan simbol pada file text}
	\label{fig:gambarDinding} 
\end{figure}

\begin{figure}[H]
	\centering  
	\includegraphics[scale=0.5]{ularDinding}  
	\caption[Ular ingin melewati jalur yang diapit oleh 2 buah dinding]{Ular ingin melewati jalur yang diapit oleh 2 buah dinding}
	\label{fig:ularDinding} 
\end{figure}


\subsection{Pengecekan tabrakan(\textit{Collision Detection})}
Pada permainan ini terdapat pengecekan tabrakan yang dapat mengecek apakah ular sudah memakan makanan, ular menabrak tubuhnya sendiri, dan ular menabrak dinding labirin. Seluruh pengecekan ini akan dilakukan pada setiap \textit{frame}. Pada pengecekan tabrakan pada apel dan ular, hanya perlu mengecek tabrakan antara kepala ular dengan apel. Karena jalur yang dilalui oleh kepala ular, akan selalu dilalui oleh bagian tubuh ular. Dengan kata lain, bagian tubuh ular akan mengikuti ke mana kepala ular akan bergerak. Dengan ini, tidak perlu dilakukan \textit{collision detection} antara bagian tubuh ular dengan apel. Cukup hanya dengan mengecek tabrakan antara kepala ular dengan apel saja. Untuk mengetahui terjadinya tabrakan antara ular dengan apel, maka akan dibuat daerah tabrakan pada apel. Daerah tabrakan ini digunakan untuk mengecek apakah 2 benda saling bertabrakan satu sama lain. Daerah tabrakan pada apel ditandai dengan arsiran bewarna merah yang terdapat pada Gambar~\ref{fig:apelArsir}. Namun, untuk membuat daerah tabrakan ini cukup sulit ketika mengecek adanya tabrakan antara ular dengan apel terutama pada bagian lengkungan pada apel. Karena itu, daerah tabrakan pada apel dibuat dengan menggunakan bentuk persegi seperti pada Gambar~\ref{fig:apelArsirPersegi}. Jika posisi kepala ular berada di dalam daerah tabrakan apel, maka dipastikan bahwa ular tersebut sudah memakan apel. Algoritma untuk mengecek tabrakan adalah sebagai berikut : cek apakah koordinat x dari kepala ular lebih besar dari posisi sisi kiri daerah tabrakan dan lebih kecil dari posisi sisi kanan daerah tabrakan. Kemudian cek apakah koordinat y dari kepala ular lebih besar dari posisi sisi atas daerah tabrakan dan lebih kecil dari posisi sisi bawah daerah tabrakan. Jika posisi kepala ular berada memenuhi ketentuan tersebut, maka kepala ular berada di dalam daerah tabrakan apel.\\

\begin{figure}[H]
	\centering  
	\includegraphics[scale=0.4]{apelArsir}  
	\caption[Daerah tabrakan pada apel]{Daerah tabrakan pada apel}
	\label{fig:apelArsir} 
\end{figure}

\begin{figure}[H]
	\centering  
	\includegraphics[scale=0.4]{apelArsirPersegi}  
	\caption[Daerah tabrakan berbentuk persegi pada apel]{Daerah tabrakan berbentuk persegi pada apel}
	\label{fig:apelArsirPersegi} 
\end{figure}

Untuk mengecek tabrakan antara ular dengan tubuhnya sendiri adalah dengan mengecek tabrakan antara kepala ular dengan seluruh bagian tubuh ular. Algoritma pengecekanya adalah sebagai berikut : jika koordinat x kepala ular lebih kecil dari koordinat x bagian tubuh ular dikurangi panjang dari bagian tubuh ular dan lebih besar dari koordinat x bagian tubuh ular ditambah dengan panjang dari bagian tubuh ular. Kemudian dicek apabila koordinat y kepala ular lebih kecil dari koordinat y bagian tubuh ular dikurangi panjang dari bagian tubuh ular dan lebih besar dari koordinat y bagian tubuh ular ditambah dengan panjang dari bagian tubuh ular. Apabila posisi kepala ular memenuhi ketentuan tersebut, maka posisi kepala ular berada di dalam daerah tabrakan pada sebuah bagian tubuh ular. Untuk mengecek tabrakan dengan labirin, lakukan pengecekan antara kepala ular dengan setiap dinding ular. \\

\section{Analisis Berorientasi Objek}

\subsection{Skenario Permainan}
Pada bagian ini akan dijelaskan dan ditunjukkan diagram \textit{use case} dari permainan \textit{Snake} 360. Penjelasan meliputi skenario, aktor, prakondisi skenario normal dan eksepsi. Aktor yang melakukanya adalah pemain. Pada Gambar~\ref{fig:useCase} terdapat diagram \textit{use case} dari permainan \textit{Snake} 360.

\begin{figure}[H]
	\centering  
	\includegraphics[scale=0.4]{useCase}  
	\caption[Diagram \textit{use case} dari permainan \textit{Snake} 360]{Diagram \textit{use case} dari permainan \textit{Snake} 360}
	\label{fig:useCase} 
\end{figure}

Berikut adalah skenario dari diagram \textit{use case} :

\begin{enumerate}
	\item Skenario : Mulai bermain \\
Aktor : Pemain \\
Prakondisi : Pemain memulai permainan.\\
Skenario normal : Pemain memulai bermain. Setelah memilih, pemain akan memilih level dan labirin. \\
Eksepsi : - \\

	\item Skenario : Memilih level dan labirin \\
Aktor : Pemain \\
Prakondisi : Pemain sudah mulai bermain. \\
Skenario normal : Pemain memilih level dan labirin yang diinginkan. \\ 
Eksepsi : - \\
\end{enumerate}

\subsection{Diagram Kelas}
Pada Gambar~\ref{fig:classDiagram} terdapat diagram kelas dari \textit{Snake} 360.

\begin{figure}[H]
	\centering  
	\includegraphics[scale=0.4]{classDiagram}  
	\caption[Diagram class dari permainan \textit{Snake} 360]{Diagram kelas dari permainan \textit{Snake} 360}
	\label{fig:classDiagram} 
\end{figure}

Diagram kelas terdiri dari beberapa kelas yaitu :

\begin{enumerate}
	\item Kelas Snake merupakan  kelas yang merepresentasikan objek ular.
	\item Kelas Apple merupakan kelas yang merepresentasikan objek apel.
	\item Kelas Game merupakan kelas yang mengatur jalanya permainan.
	\item Kelas Maze merupakan kelas yang merepresentasikan objek labirin.
	\item Kelas DrawingObject merupakan kelas untuk menggambar semua objek pada canvas.
\end{enumerate}

Berikut adalah atribut yang dimiliki setiap kelas :

\begin{enumerate}
	\item Kelas Snake\\
\textbf{int}

\begin{itemize}
	\item x, merupakan posisi ular pada koordinat x.
	\item y, merupakan posisi ular pada koordinat y.
	\item tubuhMax, merupakan panjang tubuh ular.
	\item besarUlar, merupakan lebar tubuh ular.
\end{itemize}

	\item Kelas Apel\\
	\textbf{int}
	
\begin{itemize}
	\item x, merupakan posisi apel pada koordinat x.
	\item y, merupakan posisi apel pada koordinat y.
	\item besarApel, merupakan besar apel.
\end{itemize}

\item Kelas Game \\
\textbf{int}

\begin{itemize}
	\item sudut, merupakan besar sudut yang digunakan untuk ular berbelok.
	\item score, merupakan skor yang didapat pada permainan.
	\item CANVAS\_SIZE, merupakan lebar dan tinggi canvas.
	\item GRID\_SIZE, merupakan besar grid.
\end{itemize}

\textbf{boolean}
\begin{itemize}
	\item gameOver, memberitahu apakah permainan sudah berakhir atau belum.
\end{itemize}

\item Kelas Maze \\
\textbf{int}

\begin{itemize}
	\item besarDinding, merupakan besar lebar dinding.
\end{itemize}

\end{enumerate}